\section{Derivation of the variational bound}

We wish to approximate the marginal likelihood:
\begin{equation}
\label{marginalLikelihoodSuppl}
p(Y | \bft) =  \int p( Y , F, X| \bft) \, \mathrm{d} X \,\mathrm{d}F,
\end{equation}
by computing a lower bound:
\begin{align}
\mathcal{F}_v(q, \boldsymbol \theta) = {}& \int q(\mathit{\Theta}) \log 
		\frac{ p(Y , F , \mathit{X} | \mathbf{t})}
			 {q(\mathit{\Theta})}  \, \mathrm{d} X \,\mathrm{d}F,
% 	    \nonumber \\
% 	      = {}& \sum_{d=1}^D \int q(\Theta) \log \left( p(\bfy_d | \bff_d) p(\bff_d | X) \right) dX d \bff_d -
% 		    \int q(\Theta) \frac{p(X|\bft)}{q(\Theta)} dX
		 \label{jensens1Suppl}
\end{align}
%
This can be achieved by first augmenting the joint probability density
of our model with inducing inputs $\tilde{X}$ along with their
corresponding function values $U$:
\begin{equation}
 \label{augmentedJointSuppl}
p(Y,F, U,X,\tilde{X} | \bft) = \prod_{d=1}^D p(\mathbf{y}_d | \mathbf{f}_d) p(\mathbf{f}_d | \mathbf{u}_d, \mathit{X})
p(\bfu_d | \tilde{X})  p(X | \mathbf{t})
\end{equation}
where $p(\bfu_d | \tilde{X}) = \prod_{d=1}^D \mathcal{N} \left( \bfu_d
  | \mathbf{0}, K_{MM} \right)$ . For simplicity, $\tilde{X}$ is
dropped from our expressions for the rest of this supplementary
material. Note that after including the inducing points, $p(\bff_d |
\bfu_d, X)$ remains analytically tractable and it turns out to be
\cite{rasmussen-williams}):
\begin{equation}
 \label{priorF2Suppl}
p(\bff_d | \bfu_d, X) =  \mathcal{N}  \left( \bff_d | K_{NM} K_{MM}^{-1} \bfu_d , K_{NN} - K_{NM} K_{MM}^{-1} K_{MN} \right).
\end{equation}
We are now able to define a variational distribution $q(\Theta)$ which
factorises as: For tractability we now define a variational density,
$q(\Theta)$:
\begin{equation}
\label{varDistrSuppl}
q(\mathit{\Theta}) = q(F, U,X) = q(F | U, X) q(U) q(X) = \prod_{d=1}^D p(\bff_d | \bfu_d, X )q(\bfu_d) q(X),
\end{equation}
%
%
where $q(X) = \prod_{q=1}^Q \mathcal{N} \left( \bfx_q | \bfmu_q, S_q \right)$. 
%
Now, we return to \ref{jensens1Suppl} and replace the joint
distribution with its augmented version \eqref{augmentedJointSuppl}
and the variational distribution with its factorised version
\eqref{varDistrSuppl}:
\begin{align}
\mathcal{F}_v(q, \boldsymbol \theta) = {}& \int q(\mathit{\Theta}) \log 
		\frac{ p(Y,F, U,X | \bft)}
			 {q(F, U,X)}  \, \mathrm{d} X \,\mathrm{d}F,
 	    \nonumber \\
= {}& \int \prod_{d=1}^D p(\bff_d | \bfu_d, X )q(\bfu_d) q(X) 
	    \log  \frac{\prod_{d=1}^D p(\mathbf{y}_d | \mathbf{f}_d) \cancel{p(\mathbf{f}_d | \mathbf{u}_d, \mathit{X})}
						p(\bfu_d | \tilde{X})  p(X | \mathbf{t})}
 	      		   {\prod_{d=1}^D \cancel{p(\bff_d | \bfu_d, X )}q(\bfu_d) q(X))}   \, \mathrm{d} X \,\mathrm{d}F \nonumber \\
= {}& \int \prod_{d=1}^D p(\bff_d | \bfu_d, X )q(\bfu_d) q(X) 
		\log  \frac{\prod_{d=1}^D p(\mathbf{y}_d | \mathbf{f}_d) p(\bfu_d | \tilde{X})}
				   {\prod_{d=1}^D q(\bfu_d) q(X))}   \, \mathrm{d} X \,\mathrm{d}F, \nonumber \\
- {}& \int \prod_{d=1}^D  q(X)   \log \frac{q(X)}{p(X | \mathbf{t})}   \, \mathrm{d} X \nonumber \\
= {}& \hat{\mathcal{F}}_v - \text{KL}(q \parallel p), \label{jensens1Suppl}
\end{align}
%
with $\hat{\mathcal{F}}_v =\int q(X) \log p( Y | F ) p( F | X) \,
\mathrm{d} X \,\mathrm{d}F = \sum_{d=1}^D \hat{\mathcal{F}}_d$. Both
terms in \eqref{jensens1Suppl} are analytically tractable, with the
first having the same analytical solution as the one derived in
\cite{BayesianGPLVM}.

\par
The complete form of the jensen's lower bound turns out to be:
\begin{align}
\mathcal{F}_v(q, \boldsymbol \theta) = {}& \sum_{d=1}^{D} 
	\hat{\mathcal{F}}_d(q, \boldsymbol \theta) -  \text{KL}(q \parallel p) \nonumber \\
	= {}& 
	\sum_{d=1}^{D} 
		\log \left( 
		\frac{(\beta)^{\frac{N}{2}} \vert \mathit{K_{MM}} \vert ^\frac{1}{2} }
			 {(2\pi)^{\frac{N}{2}} \vert \beta \Psi_2 + \mathit{K_{MM}}  \vert ^\frac{1}{2} } 	
		 e^{-\frac{1}{2} \mathbf{y}^{\top}_{d} W \mathbf{y}_d} 
		 \right) -
		 \frac{\beta \psi_0}{2} + \frac{\beta}{2} 
		 \text{Tr} \left( \mathit{K_{MM}^{-1}} \Psi_2 \right)  \nonumber \\
{}&	- \frac{Q}{2} \log \vert \mathit{K_t} \vert - \frac{1}{2} \sum_{q=1}^{Q}
	  \left[ \text{Tr} \left( \mathit{K_t}^{-1} \mathit{S_q} \right)	  
	  	   + \text{Tr} \left( \mathit{K_t}^{-1} \boldsymbol \mu_q \boldsymbol \mu_q^\top \right) \right] 
	 + \frac{1}{2} \sum_{q=1}^Q \log \vert \mathit{S_q} \vert + const  \label{boundFinal}
\end{align}
where the last line corresponds to the KL term. Also:
\begin{equation}
\label{psis}
\Psi_0 = \text{Tr}(\langle \mathit{K_{NN}} \rangle_{q(\mathit{X})}) \;, \;\;
\Psi_1 = \langle \mathit{K_{NM}} \rangle_{q(\mathit{X})} \;, \;\;
\Psi_2 = \langle \mathit{K_{MN}} \mathit{K_{NM}} \rangle_{q(\mathit{X})}
\end{equation}
The $\Psi$ quantities can be computed analytically as in \cite{BayesianGPLVM}.


%-------------------------

\section{Derivatives of the variational bound}
Before giving the expressions for the derivatives of the variational
bound \eqref{jensens1Suppl}, recall that the variational parameters
$\mu_q$ and $S_q$ (for all $q$s) have been reparametrised as $S_q =
\left( \mathit{K}_t^{-1} + \text{diag}(\boldsymbol \lambda_q)
\right)^{-1} \text{ and } \boldsymbol \mu_q = K_t \bar{\boldsymbol
  \mu}_q$, where the function $\text{diag}(\cdot)$ transforms a vector
into a square diagonal matrix and vice versa. Given the above, the set
of the parameters to be optimised is $( \boldsymbol \theta_f,
\boldsymbol \theta_x, \{ \bar{\bfmu}_q, \boldsymbol \lambda_q
\}_{q=1}^Q, \tilde{X}$. The gradient w.r.t. the inducing points
$\tilde{X}$, however, has exactly the same form as for $\boldsymbol
\theta_f$ and, therefore, is not presented here. Also notice that from
now on we will often use the term ``variational parameters'' to refer
to only the quantities $\bar{\bfmu}_q$ and $\boldsymbol \lambda_q$.

\textbf{Some more notation:} 
\begin{enumerate}
\item $\lambda_q$ is a scalar, an element of the vector $\boldsymbol
  \lambda_q$ which, in turn, is the main diagonal of the diagonal
  matrix $\Lambda_q$.
%\item$\lambda_m \triangleq \boldsymbol \lambda_{q;m}$, i.e. the $m$-th element of the vector $\boldsymbol \lambda_q$ (thus, an instantiation of $\lambda_q$)
\item $S_{ij} \triangleq S_{q;ij}$ the element of $S_q$ found in the
  $i$-th row and $j$-th column.
\item $\mathbf{s}_q \triangleq \lbrace S_{q;ii} \rbrace_{i=1}^N$,
  i.e. it is a vector with the diagonal of $S_q$.
%\item $s_i$ is the $i$-th element of $\mathbf{s}_q$.
\item $\text{diag}(\mathbf{s}_q)$ is a matrix full of zeros apart from
  the main diagonal which contains the vector $\mathbf{s}_q$.
\end{enumerate}

\subsection{Derivatives w.r.t. the Variational Parameters}
\begin{equation}
    \label{derivVarParamSuppl}
\frac{\vartheta \mathcal{F}_v}{\vartheta \bar{\boldsymbol \mu}_q} 
=  K_t \left( \frac{\vartheta \hat{\mathcal{F}}}{\vartheta \boldsymbol \mu_q} - \bar{\boldsymbol \mu}_q \right)
\text{ and }
 \frac{\vartheta \mathcal{F}_v}{\vartheta \boldsymbol \lambda_q}
= - ( S_q \circ S_q) \left( \frac{\vv \hat{\mathcal{F}}}{\vv \mathbf{s}_q} + \frac{1}{2} \boldsymbol \lambda_q \right).
\end{equation}

where:

\begin{align}
 \frac{\hat{\mathcal{F}}(q, \boldsymbol \theta)}{\vartheta \mu_q}
{}& = - \frac{\beta D}{2} \frac{\vartheta \Psi_0}{\vartheta \mu_q}
    + \beta \text{Tr} \left(\frac{\vartheta \Psi_1^\top}{\vartheta \mu_q} Y Y^\top \Psi_1 A^{-1} \right) \nonumber \\
{}& + \frac{\beta}{2} \text{Tr} \left[ \frac{\vartheta \Psi_2}{\vartheta \mu_q}
       \left(
	  D K_{MM}^{-1} - \beta^{-1} D A^{-1} - A^{-1} \Psi_1^\top Y Y^\top \Psi_1 A^{-1}
       \right) \right] \label{derivFTildeEfficientComputationMu}
\end{align}


\begin{align}
 \frac{\vv \hat{\mathcal{F}}(q, \boldsymbol \theta)}{\vartheta S_{q;i,j}}
{}& = - \frac{\beta D}{2} \frac{\vartheta \Psi_0}{\vartheta S_{q;i,j}}
    + \beta \text{Tr} \left(\frac{\vartheta \Psi_1^\top}{\vartheta S_{q;i,j}} Y Y^\top \Psi_1 A^{-1} \right) \nonumber \\
{}& + \frac{\beta}{2} \text{Tr} \left[ \frac{\vartheta \Psi_2}{\vartheta S_{q;i,j}}
       \left(
	  D K_{MM}^{-1} - \beta^{-1} D A^{-1} - A^{-1} \Psi_1^\top Y Y^\top \Psi_1 A^{-1}
       \right) \right] \label{derivFTildeEfficientComputationS}
\end{align}


with $A=\beta^{-1}K_{MM}+\Psi_2$.


%-------



\subsection{Derivatives w.r.t. $\boldsymbol \theta = (\boldsymbol \theta_f, \boldsymbol \theta_x)$ and $\beta$}
Given that the KL term involves only the temporal prior, its gradient w.r.t. the parameters $\boldsymbol \theta_f$ is zero. Therefore:
\begin{equation}
   \label{DerivativeOfFComplete}
      \frac{\vartheta \mathcal{F}_v}{\vartheta \theta_f} = \frac{\vartheta \hat{\mathcal{F}}}{\vartheta \theta_f}
\end{equation}

  with:

\begin{align}
\frac{\vartheta \hat{\mathcal{F}}}{\vartheta \theta_f} {}& = \text{const} - 
\frac{\beta D}{2} \frac{\vartheta \Psi_0}{\vartheta \theta_f}
 + \beta \text{Tr} \left(\frac{\vartheta \Psi_1^\top}{\vartheta \theta_f} Y Y^\top \Psi_1 A^{-1} \right) \nonumber \\
{}& + \frac{1}{2} \text{Tr} \left[ \frac{\vartheta K_{MM}}{\vartheta \theta_f}
        \left(
	   D K_{MM}^{-1} - \beta^{-1} D A^{-1} - A^{-1} \Psi_1^\top Y Y^\top \Psi_1 A^{-1} - \beta D K_{MM}^{-1} \Psi_2 K_{MM}^{-1} 
         \right) \right] \nonumber \\
{}& + \frac{\beta}{2} \text{Tr} \left[ \frac{\vartheta \Psi_2}{\vartheta \theta_f} \;\;\;\;
       \left(
	  D K_{MM}^{-1} - \beta^{-1} D A^{-1} - A^{-1} \Psi_1^\top Y Y^\top \Psi_1 A^{-1}
       \right) \right] \label{DerivativeOfFtildeComplete}
\end{align}

The expression above is identical for the derivatives w.r.t the
inducing points.  For the gradients w.r.t. the $\beta$ term, we have a
similar expression:



\begin{align}
\frac{\vartheta \hat{\mathcal{F}}}{\vartheta \beta} ={}&
  \frac{1}{2} \Big[ 
      D \left( \text{Tr}(K_{MM}^{-1} \Psi_2) + (N-M)\beta^{-1} - \Psi_0 \right) - \text{Tr}(Y Y^\top)
	  + \text{Tr}(A^{-1}\Psi_1^\top Y Y^\top \Psi_1) \nonumber \\
   +{}& \beta^{-2} D \text{Tr} ( K_{MM} A^{-1} ) + \beta^{-1} \text{Tr} \left( K_{MM}^{-1} A^{-1} \Psi_1^\top Y Y^\top \Psi_1 A^{-1} \right) \Big]
\label{derivb2}
\end{align}


In contrast to the above, the term $\hat{\mathcal{F}}_v$ does involve parameters $\boldsymbol \theta_x$, because it involves the variational parameters that are now reparametrized with $K_t$, which in turn depends on $\boldsymbol \theta_x$. 
To demonstrate that, we will forget for a moment the reparametrization of $S_q$ and we will express the bound as $F(\boldsymbol \theta_x, \mu_q (\boldsymbol \theta_x))$ (where $\mu_q (\boldsymbol \theta_x) = K_t \bar{\boldsymbol \mu_q}$) so as to show explicitly the dependency on the variational mean which is now a function of $\boldsymbol \theta_x$. Our calculations must now take into account the term
$
\left( \frac{\vartheta \hat{\mathcal{F}}(\boldsymbol \mu_q)}{\vartheta \boldsymbol \mu_q} \right)^\top
       \frac{\vartheta \mu_q (\boldsymbol \theta_x)}{\vartheta \boldsymbol \theta_x}
$
that is what we "miss" when we consider $\mu_q(\boldsymbol \theta_x) = \boldsymbol \mu_q$:
\begin{align}
\frac{\vartheta \mathcal{F}_v(\boldsymbol \theta_x, \mu_q(\boldsymbol \theta_x))}{\vartheta \theta_x} = {}&
	\frac{\vartheta \mathcal{F}_v(\boldsymbol \theta_x, \boldsymbol \mu_q)}{\vartheta \theta_x} 
  +  \left( \frac{\vartheta \hat{\mathcal{F}}(\boldsymbol \mu_q)}{\vartheta \boldsymbol \mu_q} \right)^\top
            \frac{\vartheta \mu_q(\boldsymbol \theta_x)}{\vartheta \theta_x} \nonumber \\
= {}&
 \cancel{
    \frac{\vartheta \hat{\mathcal{F}}(\boldsymbol \mu_q)}{\vartheta \theta_x}
  } +
  \frac{\vv (-\text{KL})(\boldsymbol \theta_x, \boldsymbol \mu_q(\boldsymbol \theta_x))}{\vartheta \theta_x}
+  \left( \frac{\vartheta \hat{\mathcal{F}}(\boldsymbol \mu_q)}{\vartheta \boldsymbol \mu_q} \right)^\top
            \frac{\vartheta \mu_q(\boldsymbol \theta_x)}{\vartheta \theta_x}
\label{meanReparamDerivFTheta}
\end{align}

We do the same for $S_q$ and then we can take the resulting equations and replace $\bfmu_q$ and $S_q$ with their equals so as to take the final expression which only contains $\bar{\bfmu}_q$ and $\boldsymbol \lambda_q$:

\begin{align}
\frac{\vartheta \mathcal{F}_v(\boldsymbol \theta_x, \mu_q(\boldsymbol \theta_x), S_q(\boldsymbol \theta_x))}{\vartheta \theta_x}
={}& \text{Tr} \bigg[
\Big[ - \frac{1}{2} \left( \hat{B}_q K_t \hat{B}_q + \bar{\bfmu}_q \bar{\bfmu}_q^\top \right) \nonumber \\
+{}& \left( I - \hat{B}_q K_t \right)
 diag \left(  \frac{\vv \hat{\mathcal{F}}}{\vv \mathbf{s}_q} \right)
			 \left( I - \hat{B}_q K_t \right)^\top \Big]
			  \frac{\vv K_t}{\vv \theta_x} \bigg] 	\nonumber \\	
+{}&  \left( \frac{\vartheta \hat{\mathcal{F}}( \boldsymbol \mu_q)}{\vartheta \boldsymbol \mu_q} \right)^\top
					\frac{\vv K_t}{\vv \theta_x} \bar{\boldsymbol \mu}_q 
\label{CompleteBoundDerivThetatB}
\end{align}
where $\hat{B}_q = \Lambda_q^{\frac{1}{2}} \widetilde{B}_q^{-1} \Lambda_q^{\frac{1}{2}}$.
and $\tilde{B}_q = I + \Lambda_q^{\frac{1}{2}} K_t \Lambda_q^{\frac{1}{2}}$. Note that by using this
$\tilde{B}_q$ matrix (which has eigenvalues bounded below by one) we have an expression which, when implemented, leads to more numerically stable computations, as explained in \cite{rasmussen-williams} page 45-46. 




\section{Predictions}


\section{Additional results from the experiments}
